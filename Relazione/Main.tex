\documentclass[10pt,a4paper,oneside]{report}
\usepackage[utf8]{inputenc}
\usepackage[italian]{babel}
\usepackage{amsmath}
\usepackage{amsfonts}
\usepackage{amssymb}
\usepackage{listings}
\bibliographystyle{IEEEtran}

\lstset{basicstyle=\small\ttfamily, frame=lines,
emph={obj,int,bool,State,Top,Transitions,Class,is,True,False,null,Vars,Signals},
emphstyle=\bfseries
}
\usepackage{graphicx}
\author{Luca Baldesi, Luca Melis, Simone Macaluso}
\title{Modellazione di un sistema di
Interlocking Ferroviario Distribuito
tramite UML Model Checker
}

\begin{document}
\maketitle
\tableofcontents
\chapter*{Introduzione}
In questo elaborato ci prefiggiamo di modellare un sistema di \emph{Interlocking Ferroviario} e di testarne le proprietà di safety e di stabilizzazione. L'automa e il model checking per tali caratteristiche saranno sviluppati tramite UML Model Checker \footnote{http://fmtlab.isti.cnr.it/umc/V3.9/umc.html}, un software per lo studio accademico di model based development. Il nostro lavoro si baserà su quello preesistente descritto in \cite{Paolieri} e in \cite{RossettoRocciolo}. In particolare introdurremo nel modello i semafori come ulteriore raffinamento al sistema di interlocking e proporremo dei casi di test e delle configurazioni della rete di complessità accresciuta.

Il resto del documento è composto come segue:
Nel capitolo \ref{cap:interlocking} viene presentato il problema dell'interlocking distribuito in ambito ferroviario con la specifica di come un treno può prenotare un percorso nel nostro modello. Nel capitolo \ref{cap:modelling} viene descritto il modello da noi implementato e nel capitolo \ref{cap:properties} le proprietà che abbiamo verificato per il nostro modello.

Nelle appendici \ref{cap:umc} e \ref{cap:code} vengono riportate le informazioni relative al particolare strumento di model checking che abbiamo usato e il codice relativo completo del nostro modello.
\chapter{Interlocking ferroviario}
\label{cap:interlocking}
%Che cos'è l'interlocking
Tramite un sistema di \textit{interlocking ferroviario} è possibile gestire una parte di tracciato su una rete ferroviaria e quindi permettere ai treni di effettuare richieste di prenotazione dei percorsi.\\
A differenza di un sistema centralizzato, in cui un unico centro di elaborazione si occupa di inviare i comandi ad ogni sezione del tracciato, in un sistema di \textit{interlocking ferroviario distribuito} ogni elemento della rete è in grado di eseguire in modo indipendente i propri compiti e di collaborare con i nodi adiacenti nel percorso richiesto dal treno.
\section{Elementi del tracciato}
E' possibile vedere una rete ferroviaria come un grafo $G =(V,E) $ dove i vertici $V$ sono i nodi del tracciato e gli archi $E$ sono costruiti tramite una relazione di adiacenza $R \subseteq V \times V$. in cui:
\begin{itemize}
\item un arco $e=(v,v')$ assume che $v$ sia un adiacente sinistro di $v'$
\item un arco $e=(v',v'')$ assume che $v''$ sia un adiacente destro di $v'$
\end{itemize}
Un nodo della rete puo' essere:
\begin{itemize}
\item \textbf{Circuito di Binario} è un tratto di binario e può avere soltanto un adiacente destro e un adiacente sinistro. Un circuito di binario può essere una stazione di fermata del treno;
\item \textbf{Scambio} è un punto di diramazione o di confluenza. A seconda dell' orientazione fisica, uno scambio può essere \textit{normale} o \textit{rovescio};
\item \textbf{Segnale} o \emph{Semaforo} rappresenta un segnale di protezione dell' unico nodo ad esso adiacente di tipo \emph{circuito di binario}.
\end{itemize}
Su tale grafo non esistono cicli o cappi ( nodi che hanno se stessi come adiacenti ).
Un \emph{itinerario} sul tracciato può quindi essere visto come un percorso nel grafo della rete vincolato ad avere il primo e ultimo nodo di tipo \emph{circuito di binario}.
\section{Funzionalità del sistema di Interlocking}
Le funzioni fornite dal sistema di \emph{Interlocking} sono :
\begin{itemize}
\item prenotazioni di uno o più itinerari;
\item cancellazione di itinerari prenotati;
\item liberazione dell' itinerario dopo che il treno è transitato;
\item liberazione del tracciato all'arrivo del treno nelle stazioni terminali.
\end{itemize}
\paragraph*{Prenotazione Itinerario}
La prenotazione dell' itinerario si effettua tramite il protocollo \emph{Linear Two Phase Commit Protocolo } (2PC) in cui ogni nodo della rete facente parte dell'itinerario effettua un doppio scambio di messaggi.\\
Ogni nodo conosce il suo successore e il suo predecessore lungo il percorso associato all'itinerario richiesto ed inoltre, il primo e l'ultimo nodo conoscono la loro posizione nel percorso.\\
La richiesta dell'itinerario viene effettuata dal treno tramite un messaggio di ``request'' inoltrato direttamente al primo nodo (vincolato a essere un circuito di binario). Ogni nodo del percorso, se libero, diviene ``reserved'' e propaga la \emph{request} al nodo successore fino all'ultimo nodo il quale invia a ritroso un messaggio di ``acknowledge'' che, se tutti i nodi dell'itinerario sono disponibili , sarà propagato fino al primo nodo.\\
A seguito del messaggio di ``ack'', il primo nodo invierà un ``commit'' che percorrerà nuovamente il cammino e sarà riscontrato da un messaggio di ``agree'' emesso dall'ultimo nodo.\\
Alla ricezione del messaggio di ``agree'' , il primo nodo comunica al treno la disponibilità dell'itinerario permettendo quindi al treno di iniziare a muoversi e di arrivare a destinazione.\\

Un nodo non libero rigetta la ``request'' con un messaggio di ``negative acknowledge'' che provoca la liberazione del nodo che lo riceve e che si incarica, a sua volta, di propagarlo.

\paragraph*{Cancellazione Itinerario}
La fase relativa alla cancellazione di un itinerario è stata concepita come una versione a singola fase del 2PC, in quanto i nodi risultano già essere in stato ``reserved''.\\
Il treno che intende cancellare un itinerario invia un messaggio di ``abort''. I nodi che ricevono tale messaggio si portano nello stato di ``aborting'' e inoltrano il messaggio al nodo successore, fino all'ultimo nodo del percorso.\\
L'ultimo nodo del percorso diviene libero e inoltra un messaggio di ``cancel'' in senso inverso provocando la liberazione di tutti gli altri nodi dell' itinerario. Quando il messaggio di cancel raggiunge il primo nodo, l'itinerario risulta essere cancellato.

\paragraph*{Liberazione Itinerario}
Una volta che il treno è transitato su un nodo della rete, quest' ultimo viene subito liberato per permettere la prenotazione di nuovi itinerari passanti per quel nodo.\\
\paragraph*{Liberazione tracciato}
Nel momento in cui un treno giunge nell' ultimo nodo del suo percorso (rappresentato da un circuito di binario), si deve poter far uscire il treno stesso dal tracciato.
In particolare, tale soluzione permette al treno di poter entrare in un altro tracciato.\\
Per poter fare ciò, i circuiti di binario esterni possono, tramite il comando ``free'', permettere la liberazione del tracciato consentendo quindi al treno di muoversi verso l'esterno.



\chapter{Modellazione del sistema}
\label{cap:modelling}
%come l'interlocking è stato implementato: scelte implementative
% nostro apporto al modello

\chapter{Safety e Stabilizzazione}
\label{cap:properties}
%Descrizione delle proprietà da garantire
Il sistema di interlocking distribuito da noi realizzato soddisfa la proprietà della \emph{stabilizzazione}.\\
La proprietà di \emph{stabilizzazione} può essere così enunciata:
per ogni richiesta di itinerario compiuta da un treno posto al punto di inizio dell’itinerario, esiste una computazione che porta il treno al punto di fine dell’itinerario, e tutte le computazioni alternative (per effetto di guasti o altri motivi) producono un \emph{abort} della richiesta.\\
\\
In \emph{Computation Tree Logic (CTL)} questa proprietà viene descritta come:\\
\begin{lstlisting}
AG (locomotive_request implies AF (locomotive_arrived or locomotive_cancelled))
\end{lstlisting}

In particolare:
\begin{itemize}
\item \verb!locomotive_request! rappresenta la condizione di richiesta di itinerario di un treno;
\item \verb!locomotive_arrived! rappresenta la condizione di arrivo del treno al punto di fine itinerario;
\item \verb!locomotive_cancelled! rappresenta la condizione di \emph{abort} della richiesta di itinerario.
\end{itemize}

Tutte le condizioni specificate sono state implementate mediante il meccanismo delle astrazioni sugli stati, fornito dallo strumento di modellazione \emph{UMC}.\\
Il modello sviluppato, oltre a garantire la \emph{stabilizzazione}, soddisfa anche la importante proprietà della \emph{Safety}. Tale proprietà viene soddisfatta quando:
\begin{itemize}
\item non vi sono incidenti tra treni, overo essi non si trovano contemporaneamente nello stesso circuito di binario;
\item i treni non devono trovarsi sugli scambi durante la fase di posizionamento di questi ultimi;
\item un treno deve arrivare sempre a destinazione.
\end{itemize}

\chapter{Simulazione}
\label{cap:simulation}
% casi di test trattati
La simulazione del modello è stata eseguita tramite una richiesta di itinerario da parte di un treno.\\ Come si può notare in figura \ref{lst:itinerario}, in questa richiesta vengono esercitati tutti i possibili 15 itinerari, opportunamente concatenati di modo tale che l'ultimo nodo di ciascun itinerario intermedio coincida con il primo nodo dell'itinerario successivo.\\
Il test effettuato mostra che, su questa complessa configurazione, la proprieta di \emph{stabilizzazione} viene soddisfatta.

\begin{lstlisting}[numbers=left,
numberstyle=\tiny,caption={Simulazione di itinerario effettuata},
label=lst:itinerario]
GA1 A W1 GA2 N1 W2 GA4
GA4 F W2 GA2 P1 W1 GA1
GA1 A W1 GA3 N2 W3 W2 GA4
GA4 F W2 W3 GA3 P2 W1 GA1
GA1 A W1 GA2
GA2 P1 W1 GA1
GA1 A W1 GA3
GA3 P2 W1 GA1
GA1 A W1 GA3 N2 W3 GA5
GA5 S10 W3 GA3
GA3 N2 W3 W2 GA4
GA4 F W2 GA2
GA2 N1 W2 GA4
GA4 F W2 W3 GA3
GA3 N2 W3 GA5
\end{lstlisting}

\chapter{Conclusioni}
\label{cap:conclusions}
In conclusione il modello da noi sviluppato riesce a permettere la prenotazione di un itinerario da parte di uno o più treni e gestire il sistema di segnalazione (i semafori) in maniera coerente. Inoltre viene garantita l'assenza di incidenti tra più treni. 

I possibili itinerari sono stati esaustivamente testati concatenandoli tra loro e usando un apposito script per la generazione automatica delle configurazione da sottoporre al model checker. Sia in \cite{Paolieri} che in \cite{RossettoRocciolo} si lamentava la mancanza di un tale tool per rendere agevole la fase di test e stesura di configurazione di una eventuale rete ferroviaria diversa.

\appendix

\chapter{UML Model Checker}
\section{pippo}
%copia spudorata
\chapter{Codice del modello}
\label{cap:code}
\label{cap:code}
\begin{lstlisting}[caption={modello della classe Train}]
Class Train is

  Signals:
    start, wait, aborted, retry;

  Vars:
    routes: int[];
    route_pos: int := 0;
    nodes: obj[];
    position: int := 0;
    stops: obj[];
    stop_pos: int := 0;
    node: obj;
    aborting: bool[];

  State Top = READY, WAITING_OK, MOVING, ARRIVED, STOPPED, ABORTING, CANCELLED

  Transitions:
    READY -> WAITING_OK { - / 
    nodes[position].req(self, nodes[position], routes[route_pos]); }

    WAITING_OK -> CANCELLED { wait }  
    WAITING_OK -> MOVING { start }
    WAITING_OK -> ABORTING { - [aborting[position] = True] / 
    nodes[position].aborting(self, nodes[position], routes[route_pos]); }
    
    CANCELLED -> READY

    MOVING -> ABORTING { - [aborting[position] = True 
    and position < nodes.length - 1 and stops[stop_pos] /= nodes[position]] / 
    nodes[position].aborting(self, nodes[position], routes[route_pos]); }
    
    MOVING -> MOVING { - [aborting[position] = False 
    and position < nodes.length - 1 
    and stops[stop_pos] /= nodes[position]] /                    
                   node := null; --perche' queste azioni non sono atomiche!
                   -- sempre perche' non sono operazioni atomiche
                   -- il treno si trova per un istante su due nodi cosi da
                   -- evitare una doppia prenotazione. Quindi si setta prima 
                   -- su ON il nuovo
                   -- nodo e poi su OFF il nodo che viene lasciato.
    nodes[position + 1].sensorOn(self, nodes[position + 1], routes[route_pos]);
    nodes[position].sensorOff(self, nodes[position], routes[route_pos]);
    position := position + 1;
    node := nodes[position];
    }

    MOVING -> STOPPED { - [stops[stop_pos] = nodes[position] 
    and position < nodes.length - 1] 
    / nodes[position].train_stopped(self); }
    MOVING -> ARRIVED  { - [position = nodes.length - 1] 
    / node := null; 
    node := nodes[position].sensorOff(self, nodes[position], routes[route_pos]); }
    
    STOPPED -> READY { - [position < nodes.length - 1] /
                   stop_pos := stop_pos + 1;
                   route_pos := route_pos + 1;
    }
    
    ABORTING -> READY { aborted / 
    aborting[position] := False; }
    ABORTING -> MOVING { retry / 
    aborting[position] := False; aborting[position + 1] := True; }

end Train
\end{lstlisting}
\begin{lstlisting}[caption={modello della classe TrackCircuit}]
Class TrackCircuit is

  Signals:
    req(sender: obj, dest: obj, route: int);        
    ack(sender: obj, dest: obj, route: int);
    nack(sender: obj, dest: obj, route: int);
    commit(sender: obj, dest: obj, route: int);
    agree(sender: obj, dest: obj, route: int);
    disagree(sender: obj, dest: obj, route: int);
    train_stopped(sender: obj);
    aborting(sender: obj, dest: obj, route: int);
    cancel(sender: obj, dest: obj, route: int);

  Operations:
    sensorOn(sender: obj, dest: obj, route: int);
    sensorOff(sender: obj, dest: obj, route: int);

  Vars:
    next: obj[];
    prev: obj[];
    train: obj := null;
    outer_station: bool;

  State Top = FREE, WAITING_ACK, WAITING_COMMIT, WAITING_AGREE, RESERVED, 
  TRAIN_PASSING, ABORTING

  Transitions:
    FREE -> WAITING_ACK { req(sender, dest, route) [(train=null or sender=train) 
    and next[route] /= null] /
     next[route].req(self, next[route], route); }
    FREE -> WAITING_COMMIT { req(sender, dest, route) [(train=null or sender=train) 
    and next[route] = null] / 
    prev[route].ack(self, prev[route], route); }
    FREE -> FREE { req(sender, dest, route) [train/=null and sender/=train] 
    / sender.nack(self, sender, route); }

    WAITING_ACK -> WAITING_COMMIT { ack(sender, dest, route) 
    [prev[route] /= null and train = null] / 
    prev[route].ack(self, prev[route], route); }
    WAITING_ACK -> WAITING_AGREE{ ack(sender, dest, route) [train /= null] / 
    next[route].commit(self, next[route], route); }
    WAITING_ACK -> FREE { nack(sender, dest, route_id) 
    [(prev[route] = null) 
    and (train /= null)] / 
    train.wait; }
    WAITING_ACK -> FREE { nack(sender, dest, route_id) 
    [prev[route] /= null] / 
    prev[route].nack(self, prev[route], route); }
    WAITING_ACK -> WAITING_ACK { req(sender, dest, route) / 
    sender.nack(self, sender, route); }
    WAITING_ACK -> ABORTING { aborting(sender, dest, route) / 
    next[route].aborting(self, next[route], route); }

    WAITING_COMMIT -> WAITING_AGREE{ commit(sender, dest, route) 
    [next[route] /= null] /  next[route].commit(self, next[route], route); }
    WAITING_COMMIT -> Top.RESERVED{ commit(sender, dest, route) 
    [next[route] = null] / 
    prev[route].agree(self, prev[route], route); }
    WAITING_COMMIT -> WAITING_COMMIT { req(sender, dest, route) / 
    sender.nack(self, sender, route); }
    WAITING_COMMIT -> ABORTING { aborting(sender, dest, route) 
    [next[route] /= null] / 
    next[route].aborting(self, next[route], route); }
    WAITING_COMMIT -> FREE { aborting(sender, dest, route) 
    [next[route] = null] / 
    prev[route].cancel(self, prev[route], route); }

    WAITING_AGREE -> Top.RESERVED { agree(sender, dest, route) [prev[route] /= null 
    and train = null] / prev[route].agree(self, prev[route], route); }
    WAITING_AGREE -> Top.RESERVED { agree(sender, dest, route) [train /= null] / 
    train.start; }
    WAITING_AGREE -> FREE { disagree(sender, dest, route_id) 
    [(prev[route] = null) and (train /= null)] / 
    train.wait; }
    WAITING_AGREE -> FREE { disagree(sender, dest, route_id) 
    [prev[route] /= null] / 
    prev[route].disagree(self, prev[route], route); }
    WAITING_AGREE -> WAITING_AGREE { req(sender, dest, route) / 
    sender.nack(self, sender, route); }
    WAITING_AGREE -> ABORTING { aborting(sender, dest, route) / 
    next[route].aborting(self, next[route], route); }

    Top.RESERVED -> TRAIN_PASSING { sensorOn(sender,dest,route) / 
    train := sender; }
    Top.RESERVED -> FREE { disagree(sender, dest, route) / 
    if next[route] /= null then 
    { next[route].disagree(self, next[route], route) }; }
    Top.RESERVED -> FREE { sensorOff(sender,dest,route) / 
    train := null; }
    Top.RESERVED -> Top.RESERVED { req(sender, dest, route) / 
    sender.nack(self, sender, route); }
    Top.RESERVED -> ABORTING { aborting(sender, dest, route) 
    [next[route] /= null] / 
    next[route].aborting(self, next[route], route); }
    Top.RESERVED -> FREE { aborting(sender, dest, route) 
    [next[route] = null] / 
    prev[route].cancel(self, prev[route], route); }

    TRAIN_PASSING -> FREE { sensorOff(sender,dest,route) 
    [next[route] = null] / 
    if(outer_station = True) { train := null; return null;} 
    else { return self; } }
    TRAIN_PASSING -> FREE { sensorOff(sender,dest,route) 
    [next[route] /= null] / 
    train := null; }
    TRAIN_PASSING -> FREE { train_stopped(sender) 
    [train = sender] }
    TRAIN_PASSING -> TRAIN_PASSING { req(sender, dest, route) / 
    sender.nack(self, sender, route); }
    TRAIN_PASSING -> ABORTING { aborting(sender, dest, route) / 
    next[route].aborting(self, next[route], route); }

    ABORTING -> FREE { cancel(sender, dest, route) 
    [train /= null] / 
    train.aborted; }
    ABORTING -> FREE { cancel(sender, dest, route) 
    [train = null and prev[route] /= null] / 
    prev[route].cancel(self, prev[route], route); }

end TrackCircuit

\end{lstlisting}
\begin{lstlisting}[caption={modello della classe Switch}]
Class Switch is

  Signals:
    req(sender: obj, dest: obj, route: int);        
    ack(sender: obj, dest: obj, route: int);
    nack(sender: obj, dest: obj, route: int);
    commit(sender: obj, dest: obj, route: int);
    agree(sender: obj, dest: obj, route: int);
    disagree(sender: obj, dest: obj, route: int);
    aborting(sender: obj, dest: obj, route: int);
    cancel(sender: obj, dest: obj, route: int);

  Operations:
    sensorOn(sender: obj, dest: obj, route: int);  
    sensorOff(sender: obj, dest: obj, route: int);

  Vars:
    next: obj[];             
    prev: obj[];             
    conf: bool[];
    reversed: bool := False; 
    train: obj := null;      
    requested_route: int;         

  State Top = FREE, WAITING_ACK, WAITING_COMMIT, 
  WAITING_AGREE, POSITIONING, RESERVED, 
  TRAIN_PASSING, CHECK_POSITION, ABORTING
  
  State WAITING_ACK Defers req(sender: obj, dest: obj, route: int)

  Transitions:
    FREE -> WAITING_ACK { req(sender, dest, route) / 
    next[route].req(self, next[route], route); }

    WAITING_ACK -> WAITING_COMMIT { ack(sender, dest, route) / 
    prev[route].ack(self, prev[route], route); }
    WAITING_ACK -> FREE { nack(sender, dest, route) / 
    prev[route].nack(self, prev[route], route); }
    WAITING_ACK -> WAITING_ACK { req(sender, dest, route) / 
    sender.nack(self, sender, route); }
    WAITING_ACK -> ABORTING { aborting(sender, dest, route) / 
    next[route].aborting(self, next[route], route); }

    WAITING_COMMIT -> WAITING_AGREE { commit(sender, dest, route) / 
    				next[route].commit(self, next[route], route); }
    WAITING_COMMIT -> WAITING_COMMIT { req(sender, dest, route) / 
    sender.nack(self, sender, route); }
    WAITING_COMMIT -> ABORTING { aborting(sender, dest, route) / 
    next[route].aborting(self, next[route], route); }

    WAITING_AGREE -> FREE { disagree(sender, dest, route) / 
    prev[route].disagree(self, prev[route], route); }
    WAITING_AGREE -> WAITING_AGREE { req(sender, dest, route) / 
    sender.nack(self, sender, route); }
    WAITING_AGREE -> CHECK_POSITION { agree(sender, dest, route) / 
    requested_route := route; }
    WAITING_AGREE -> ABORTING { aborting(sender, dest, route) / 
    next[route].aborting(self, next[route], route); }

    CHECK_POSITION -> Top.RESERVED { - [reversed = conf[requested_route]] / 
    prev[requested_route].agree(self, prev[requested_route], 
    requested_route); }
    CHECK_POSITION -> POSITIONING { - [reversed /= conf[requested_route]] }
    CHECK_POSITION -> CHECK_POSITION { req(sender, dest, route) / 
    sender.nack(self, sender, route); }
    CHECK_POSITION -> ABORTING { aborting(sender, dest, route) / 
    next[route].aborting(self, next[route], route); }

    POSITIONING -> Top.RESERVED { - / reversed := not reversed; 
    prev[requested_route].agree(self, prev[requested_route], 
    requested_route); }
    POSITIONING -> FREE  { - / 
    prev[requested_route].disagree(self, prev[requested_route], 
    requested_route); 
    next[requested_route].disagree(self, next[requested_route], 
    requested_route); }
    POSITIONING -> POSITIONING { req(sender, dest, route) / 
    sender.nack(self, sender, route); }
    POSITIONING -> ABORTING { aborting(sender, dest, route) / 
    next[route].aborting(self, next[route], route); }

    Top.RESERVED -> TRAIN_PASSING { sensorOn(sender,dest,route) / 
    train := sender; }
    Top.RESERVED -> FREE { disagree(sender, dest, route) / 
    next[route].disagree(self, next[route], route); }
    Top.RESERVED -> Top.RESERVED { req(sender, dest, route) / 
    sender.nack(self, sender, route); }
    Top.RESERVED -> ABORTING { aborting(sender, dest, route) / 
    next[route].aborting(self, next[route], route); }

    TRAIN_PASSING -> FREE { sensorOff(sender,dest,route) / 
    train := null; }
    TRAIN_PASSING -> TRAIN_PASSING { req(sender, dest, route) / 
    sender.nack(self, sender, route); }
    TRAIN_PASSING -> TRAIN_PASSING { aborting(sender, dest, route) 
    [train /= null and sender = train] / train.retry; }

    ABORTING -> FREE { cancel(sender, dest, route) / 
    prev[route].cancel(self, prev[route], route); }

end Switch
\end{lsrlisting}

\begilsstlisting}[caption={modello della classe Semaphore}]
Class Semaphore is

  Signals:
    req(sender: obj, dest: obj, route: int);        
    ack(sender: obj, dest: obj, route: int);
    nack(sender: obj, dest: obj, route: int);
    commit(sender: obj, dest: obj, route: int);
    agree(sender: obj, dest: obj, route: int);
    disagree(sender: obj, dest: obj, route: int);
    aborting(sender: obj, dest: obj, route: int);
    cancel(sender: obj, dest: obj, route: int);

  Operations:
    sensorOn(sender: obj, dest: obj, route: int);  
    sensorOff(sender: obj, dest: obj, route: int);

  Vars:
    next: obj[];             
    prev: obj[];             
    train: obj := null;      
    requested_route: int;
    color_red: bool := True ;          

  State Top = RED, WAITING_ACK, WAITING_COMMIT, WAITING_AGREE, 
  COMMANDED_RED, COMMANDED_GREEN, TRAIN_PASSING, GREEN, ABORTING, FAILED
  
  State WAITING_ACK Defers req(sender: obj, dest: obj, route: int)

  Transitions:
    RED -> WAITING_ACK { req(sender, dest, route) / 
    next[route].req(self, next[route], route); }

    WAITING_ACK -> WAITING_COMMIT { ack(sender, dest, route) / 
    prev[route].ack(self, prev[route], route); }
    WAITING_ACK -> RED { nack(sender, dest, route) / 
    prev[route].nack(self, prev[route], route); }
    WAITING_ACK -> WAITING_ACK { req(sender, dest, route) / 
    sender.nack(self, sender, route); }
    WAITING_ACK -> ABORTING { aborting(sender, dest, route) / 
    next[route].aborting(self, next[route], route); }

    WAITING_COMMIT -> WAITING_AGREE { commit(sender, dest, route) / 
    next[route].commit(self, next[route], route); }
    WAITING_COMMIT -> WAITING_COMMIT { req(sender, dest, route) / 
    sender.nack(self, sender, route); }
    WAITING_COMMIT -> ABORTING { aborting(sender, dest, route) / 
    next[route].aborting(self, next[route], route); }

    WAITING_AGREE -> RED { disagree(sender, dest, route) / 
    prev[route].disagree(self, prev[route], route); }
    WAITING_AGREE -> WAITING_AGREE { req(sender, dest, route) / 
    sender.nack(self, sender, route); }
    WAITING_AGREE -> COMMANDED_GREEN { agree(sender, dest, route) / 
    requested_route := route; }
    WAITING_AGREE -> ABORTING { aborting(sender, dest, route) / 
    next[route].aborting(self, next[route], route); }

    COMMANDED_GREEN -> Top.GREEN { - / 
    prev[requested_route].agree(self, 
    prev[requested_route], requested_route); 
    color_red := False; }
    COMMANDED_GREEN -> RED  { - / 
    prev[requested_route].disagree(self, 
    prev[requested_route], requested_route); 
    next[requested_route].disagree(self, 
    next[requested_route], requested_route); }
    
    COMMANDED_GREEN -> COMMANDED_GREEN { req(sender, dest, route) / 
    sender.nack(self, sender, route); }
    
    COMMANDED_GREEN -> ABORTING { aborting(sender, dest, route) / 
    next[route].aborting(self, next[route], route); }

    Top.GREEN -> TRAIN_PASSING { sensorOn(sender,dest,route) / 
    train := sender; }
    Top.GREEN -> Top.GREEN { req(sender, dest, route) / 
    sender.nack(self, sender, route); }
    Top.GREEN -> ABORTING { aborting(sender, dest, route) / 
    next[route].aborting(self, next[route], route); }

    TRAIN_PASSING -> COMMANDED_RED { sensorOff(sender,dest,route) / 
    train := null; }
    TRAIN_PASSING -> TRAIN_PASSING { req(sender, dest, route) / 
    sender.nack(self, sender, route); }
    TRAIN_PASSING -> TRAIN_PASSING { aborting(sender, dest, route) 
    [train /= null and sender = train] / train.retry; }
 
    COMMANDED_RED -> COMMANDED_RED { req(sender, dest, route) / 
    sender.nack(self, sender, route); }
    COMMANDED_RED -> FAILED
    COMMANDED_RED -> RED {- / 
    color_red := True;}
    
    FAILED -> FAILED { req(sender, dest, route) / 
    sender.nack(self, sender, route); }
    FAILED -> RED {- / 
    color_red := True;}

    ABORTING -> RED { cancel(sender, dest, route) / 
    prev[route].cancel(self, prev[route], route); }

end Semaphore
\end{lstlisting}

\begin{lstlisting}[caption={oggetti istanziati nella simulazione}]
Objects

GA1: TrackCircuit (
 train => null, 
    outer_station => True,
 prev => [null,null,null,null,null,null,W1,null,W1,null,W1,null,W1
 ,null,null],
 next => [A,A,A,A,A,null,null,null,null,null,null,null,null,null,
 null]);
A: Semaphore (
    train => null,
 prev => [GA1,GA1,GA1,GA1,GA1,null,null,null,null,null,null,null,
 null,null,null],
 next => [W1,W1,W1,W1,W1,null,null,null,null,null,null,null,null,
 null,null]);
W1: Switch (
 conf => [False, False, False, False, False, False, False, False, 
 False, False, False, False, False, True, True],
    train => null,
 prev => [A,A,A,A,A,null,P1,null,P2,null,P1,null,
 P2,null,null],
 next => [GA2,GA2,GA3,GA3,GA3,null,GA1,null,GA1,null,
 GA1,null,GA1,null,null]);
GA2: TrackCircuit (
 train => null, 
    outer_station => False,
 prev => [W1,W1,null,null,null,W2,W2,null,null,null,null,null,
 null,null,null],
 next => [null,N1,null,null,null,null,P1,null,null,null,P1,N1,
 null,null,null]);
N1: Semaphore (
    train => null,
 prev => [null,GA2,null,null,null,null,null,null,null,null,null,
 GA2,null,null,null],
 next => [null,W2,null,null,null,null,null,null,null,null,null,
 W2,null,null,null]);
W2: Switch (
 conf => [False, False, False, False, False, False, False, False, 
 False, False, False, 
 False, False, True, True],
 prev => [null,N1,null,W3,null,F,F,F,F,null,null,N1,null,W3,null],
 next => [null,GA4,null,GA4,null,GA2,GA2,W3,W3,null,null,GA4,null,
 GA4,null]);
GA4: TrackCircuit (
 train => null, 
    outer_station => True,
 prev => [null,W2,null,W2,null,null,null,null,null,null,null,W2,
 null,W2,null],
 next => [null,null,null,null,null,F,F,F,F,null,null,null,null,
 null,null]);
GA3: TrackCircuit (
 train => locomotive, 
    outer_station => False,
 prev => [null,null,W1,W1,W1,null,null,W3,W3,W3,null,null,null,
 null,null],
 next => [null,null,null,N2,N2,null,null,null,P2,null,null,null,
 P2,N2,N2]);
N2: Semaphore (
    train => null,
 prev => [null,null,null,GA3,GA3,null,null,null,null,null,null,
 null,null,GA3,GA3],
 next => [null,null,null,W3,W3,null,null,null,null,null,null,
 null,null,W3,W3]);
W3: Switch (
 conf => [False, False, False, False, False, False, False, 
 False, False, False, False, 
 False, False, True, True],
 prev => [null,null,null,N2,N2,null,null,W2,W2,S10,null,
 snull,null,N2,N2],
 next => [null,null,null,W2,GA5,null,null,GA3,GA3,GA3,null,
 null,null,W2,GA5]);
GA5: TrackCircuit (
 train => null, 
    outer_station => False,
 prev => [null,null,null,null,W3,null,null,null,null,null,null,null,
 null,null,W3],
 next => [null,null,null,null,null,null,null,null,null,S10,null,null,
 null,null,null]);
F: Semaphore (
    train => null,
 prev => [null,null,null,null,null,GA4,GA4,GA4,GA4,null,null,null,null,
 null,null],
 next => [null,null,null,null,null,W2,W2,W2,W2,null,null,null,null,null,null]);
P1: Semaphore (
    train => null,
 prev => [null,null,null,null,null,null,GA2,null,null,null,GA2,null,null,
 null,null],
 next => [null,null,null,null,null,null,W1,null,null,null,W1,null,null,null,null]);
P2: Semaphore (
    train => null,
 prev => [null,null,null,null,null,null,null,null,GA3,null,null,null,GA3,
 null,null],
 next => [null,null,null,null,null,null,null,null,W1,null,null,null,W1,
 null,null]);
S10: Semaphore (
    train => null,
 prev => [null,null,null,null,null,null,null,null,null,GA5,null,null,null,
 null,null],
 next => [null,null,null,null,null,null,null,null,null,W3,null,null,null,
 null,null]);
 
 locomotive: Train (
 	routes => [ 1,6,3,8,0,10,2,12,4,9,13,5,11,7,14],
 	nodes => [ GA1,A,W1,GA2,N1,W2,GA4,F,W2,GA2,P1,W1,GA1,A,W1,GA3,N2,W3,W2,
 	GA4,F,W2,W3,
 	GA3,P2,W1,GA1,A,W1,GA2,P1,W1,GA1,A,W1,GA3,P2,W1,GA1,A,W1,GA3,N2,W3,GA5,
 	S10,W3,GA3,
 	N2,W3,W2,GA4,F,W2,GA2,
 	N1,W2,GA4,F,W2,W3,GA3,N2,W3,GA5],
 	aborting => [False,False,False,False,False,False,False,False,False,False,
 	False,False,False,False,False,False,False,False,False,False,False,False,
 	False,False,False,False,False,False,False,False,False,False,False,False,
 	False,False,False,False,False,False,False,False,False,False,False,False,
 	False,False,False,False,False,False,False,False,
 	False,False,False,False,False,False,False,False,
 	False,False,False],
 	stops => [GA4,GA1,GA4,GA1,GA2,GA1,GA3,GA1,GA5,GA3,GA4,GA2,
 	GA4,GA3,GA5],
 	node => GA1
 );
 
\end{lstlisting}

\begin{lstlisting}[caption={Astrazioni per la verifica delle proprieta'}]
Abstractions {
  Action $1($*) -> $1($*)
    
  State inState(locomotive.ARRIVED) -> locomotive_arrived

  State inState(locomotive.CANCELLED) -> locomotive_cancelled

  State inState(locomotive.WAITING_OK) -> locomotive_request
  }
\end{lstlisting}

\bibliography{bibliography}
\end{document}


