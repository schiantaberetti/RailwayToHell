\documentclass[10pt,a4paper,oneside]{report}
\usepackage[utf8]{inputenc}
\usepackage[italian]{babel}
\usepackage{amsmath}
\usepackage{amsfonts}
\usepackage{amssymb}
\usepackage{graphicx}
\author{Luca Baldesi, Luca Melis, Simone Macaluso}
\title{Modellazione di un sistema di
Interlocking Ferroviario Distribuito
tramite UML Model Checker
}

\begin{document}
\maketitle

\chapter*{Introduzione}
\section{asdf}
\chapter{Interlocking ferroviario}
%Che cos'è l'interlocking
Tramite un sistema di \textit{interlocking ferroviario} è possibile gestire una parte di tracciato su una rete ferroviaria e quindi permettere ai treni di effettuare richieste di prenotazione dei percorsi.\\
A differenza di un sistema centralizzato, in cui un unico centro di elaborazione si occupa di inviare i comandi ad ogni sezione del tracciato, in un sistema di \textit{interlocking ferroviario distribuito} ogni elemento della rete è in grado di eseguire in modo indipendente i propri compiti e di collaborare con i nodi adiacenti nel percorso richiesto dal treno.
\section{Elementi del tracciato} 
E' possibile vedere una rete ferroviaria come un grafo $G =(V,E) $ dove i vertici $V$ sono i nodi del tracciato e gli archi $E$ sono costruiti tramite una relazione di adiacenza $R \subseteq V \times V$. in cui:
\begin{itemize}
\item un arco $e=(v,v')$ assume che $v$ sia un adiacente sinistro di $v'$ 
\item un arco $e=(v',v'')$ assume che $v''$ sia un adiacente destro di $v'$ 
\end{itemize}
Un nodo della rete puo' essere:
\begin{itemize}
\item \textbf{Circuito di Binario} è un tratto di binario e può avere soltanto un adiacente destro e un adiacente sinistro. Un circuito di binario può essere una stazione di fermata del treno;
\item \textbf{Scambio} è un punto di diramazione o di confluenza. A seconda dell' orientazione fisica, uno scambio può essere \textit{normale} o \textit{rovescio}; 
\item \textbf{Segnale} o \emph{Semaforo} rappresenta un segnale di protezione dell' unico nodo ad esso adiacente di tipo \emph{circuito di binario}.
\end{itemize}
Su tale grafo non esistono cicli o cappi ( nodi che hanno se stessi come adiacenti ).
Un \emph{itinerario} sul tracciato può quindi essere visto come un percorso nel grafo della rete vincolato ad avere il primo e ultimo nodo di tipo \emph{circuito di binario}.
\section{Funzionalità del sistema di Interlocking}
Le funzioni fornite dal sistema di \emph{Interlocking} sono :
\begin{itemize}
\item prenotazioni di uno o più itinerari;
\item cancellazione di itinerari prenotati;
\item liberazione dell' itinerario dopo che il treno è transitato; 
\item liberazione del tracciato all'arrivo del treno nelle stazioni terminali.
\end{itemize} 
\paragraph*{Prenotazione Itinerario}
La prenotazione dell' itinerario si effettua tramite il protocollo \emph{Linear Two Phase Commit Protocolo } (2PC) in cui ogni nodo della rete facente parte dell'itinerario effettua un doppio scambio di messaggi.\\
Ogni nodo conosce il suo successore e il suo predecessore lungo il percorso associato all'itinerario richiesto ed inoltre, il primo e l'ultimo nodo conoscono la loro posizione nel percorso.\\
La richiesta dell'itinerario viene effettuata dal treno tramite un messaggio di ``request'' inoltrato direttamente al primo nodo (vincolato a essere un circuito di binario). Ogni nodo del percorso, se libero,  diviene ``reserved'' e propaga la \emph{request} al nodo successore fino all'ultimo nodo il quale invia a ritroso un messaggio di ``acknowledge'' che, se tutti i nodi dell'itinerario sono disponibili , sarà propagato fino al primo nodo.\\
A seguito del messaggio di ``ack'', il primo nodo invierà un ``commit'' che percorrerà nuovamente il cammino e sarà riscontrato da un messaggio di ``agree'' emesso dall'ultimo nodo.\\
Alla ricezione del messaggio di ``agree'' , il primo nodo comunica al treno la disponibilità dell'itinerario permettendo quindi al treno di iniziare a muoversi e di arrivare a destinazione.\\

Un nodo non libero rigetta la ``request'' con un messaggio di ``negative acknowledge'' che provoca la liberazione del nodo che lo riceve e che si incarica, a sua volta, di propagarlo.

\paragraph*{Cancellazione Itinerario}
La fase relativa alla cancellazione di un itinerario è stata concepita come una versione a singola fase del 2PC, in quanto i nodi risultano già essere in stato ``reserved''.\\
Il treno che intende cancellare un itinerario invia un messaggio di ``abort''. I nodi che ricevono tale messaggio si portano nello stato di ``aborting'' e inoltrano il messaggio al nodo successore, fino all'ultimo nodo del percorso.\\
L'ultimo nodo del percorso diviene libero e inoltra un messaggio di ``cancel'' in senso inverso provocando la liberazione di tutti gli altri nodi dell' itinerario. Quando il messaggio di cancel raggiunge il primo nodo, l'itinerario risulta essere cancellato.

\paragraph*{Liberazione Itinerario}
Una volta che il treno è transitato su un nodo della rete, quest' ultimo viene subito liberato per permettere la prenotazione di nuovi itinerari passanti per quel nodo.\\
\paragraph*{Liberazione tracciato}
Nel momento in cui un treno giunge nell' ultimo nodo del suo percorso (rappresentato da un circuito di binario), si deve poter far uscire il treno stesso dal tracciato.
In particolare, tale soluzione permette al treno di poter entrare in un altro tracciato.\\
Per poter fare ciò, i circuiti di binario esterni possono, tramite il comando ``free'', permettere la liberazione del tracciato consentendo quindi al treno di muoversi verso l'esterno.



\chapter{Modellazione del sistema}
%come l'interlocking è stato implementato: scelte implementative
% nostro apporto al modello

\chapter{Safety e Stabilizzazione}
%Descrizione delle proprietà da garantire

\chapter{Simulazione}
% casi di test trattati

\chapter{Conclusioni}
\appendix

\chapter{UML Model Checker}
\section{pippo}
%copia spudorata
\chapter{Codice del modello}
\end{document}