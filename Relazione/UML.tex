\chapter{UML Model Checker}
\label{cap:umc}
Lo strumento UMC permette di effettuare il model checking di formule $\mu$UCTL
a partire da:
\begin{itemize}
  \item un insieme di statecharts  che descrivono il comportamento dinamico delle
classi del sistema;
\item un diagramma degli oggetti per lo stato iniziale del sistema;
\item una serie di “criteri” che specificano quali attributi ed eventi
osservare.
\end{itemize}

	Una formula $\mu$UCTL è verificata “on-the-fly” , generando incrementalmente
le e parti della macchina a stati necessarie alla verifica corrente.
	
	L’algoritmo di model checking impiegato è di tipo \textit{bounded} ed
effettua successive e ricerche in profondità aumentando il limite fino a
determinare se la formula è soddisfatta o meno. Questo, oltre a impedire l’esplosione dello spazio degli stati, permette di fornire controesempi di lunghezza ridotta nel caso di formule non soddisfatte.

\section{Definizione del modello}

La struttura dei modelli UMC comprende tre sezioni:

\textbf{Definizioni delle classi} Definisce le classi del sistema in termini di macchine a stati
UML. Ogni definizione di classe permette di specificare:

\begin{itemize}
	\item gli eventi asincroni (\textit{Signals}) e le chiamate sincrone
	(\textit{Operations}) accettate dalla classe;
	\item le variabili locali della classe (\textit{Vars});

	\item la struttura di stati e sottostati della statechart (\textit{State});
	
	\item le transizioni, con trigger, guardia ed azioni.
\end{itemize}

\textbf{Dichiarazione degli oggetti} Una volta definite le classi ` possibile istanziarle ed
e inizializzare gli oggetti impostandone le variabili locali. I nomi associati
alle istanze degli oggetti costituiscono variabili globali visibili all’interno di tutte le
definizioni di classe.

\textbf{Regole di astrazione} Specificano quali eventi e propriet` del sistema osservare, e
a
giocano un ruolo essenziale al momento della verifica, costituendo i predicati
atomici su cui definire formule $\mu$ UCTL.

\section{Verifica di formule in logica temporale}
UMC supporta numerose logiche temporali che comprendono tutto il $\mu$-Calcolo,
oltre ad operatori di più alto livello simili a quelli delle logiche CTL ed
\textit{Action-based CTL} (ACTL).

In particolare, sono presenti:
\begin{itemize}
  \item operatore \textit{diamond} \textless  \texttt{action}\textgreater$\phi$
  e \textit{weak diamond} \textless\textless  \texttt{action}\textgreater
  \textgreater$\phi$;
  \item operatore \textit{box} [action]$\phi$ e \textit{weak
  box}[[action]]$\phi$;
  \item operatori di massimo e minimo punto fisso (max, min);
  \item operatori EX$\{action\}\phi$, AX$\{action\}\phi$, ET$\phi$, AT$\phi$;
  \item operatori EF$\phi$, AF$\phi$, EG$\phi$, AG$\phi$;
  \item operatori E[$\phi_{1}\{act1\}\cup\phi_{2}$] e
  A[$\phi_{1}\{act1\}\cup\phi_{2}$]
  \item operatori E[$\phi_{1}\{act1\}\cup\{act2\}\phi_{2}$] e
  A[$\phi_{1}\{act1\}\cup\{act2\}\phi_{2}$]
\end{itemize}

I predicati atomici di \textit{state formulae} e \textit{action formulae} sono
definiti attraverso regole di astrazione, che specificano quali condizioni e
quali azioni osservare nel sistema, associando ad esse un nome utilizzabile
nelle formule in logica temporale.
Per la verifica è possibile utilizzare una versione a riga di comando di UMC
oppure e
l’interfaccia web disponibile all’indirizzo http://fmt.isti.cnr.it/umc [10].
L’interfaccia web dispone di numerose funzionalità aggiuntive, fra cui la
minimizzazione a dell’automa e la generazione di rappresentazioni grafiche.
Per una guida all’uso si rimanda direttamente al sito.




